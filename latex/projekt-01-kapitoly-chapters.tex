% Tento soubor nahraďte vlastním souborem s obsahem práce.
%=========================================================================
% Autoři: Michal Bidlo, Bohuslav Křena, Jaroslav Dytrych, Petr Veigend a Adam Herout 2019
\chapter{Úvod}
TODO

\chapter{Blockchain}\label{chap:blockchain}

\section{Moderná kryptografia}\label{sec:crypto}
Pre pochopenie technológie blockchain je potrebná základná znalosť modernej kryptografie. V tejto sekcii je popísaný kryptografický hash (pozri~\ref{subsec:hash}) a digitálny podpis (pozri~\ref{subsec:sign}). Obe tieto kryptografické primitíva sú základom na ktorom stojí integrita a anonymita blockchainu.

\subsection{Hash}\label{subsec:hash}
Hashovacia funkcia je taká funkcia $h$, ktorá má ako parameter $x$ reťazec bitov ľubovolnej dĺžky a vracia reťazec $y$ s konštantnou dĺžkou. Reťazec $y$ voláme hash. Hashovacia funkcia vracia pre konkrétny vstup vždy rovnaký hash.
$$ h(x) = y $$
Kryptografická hashovacia funkcia, alebo tiež jednocestná funkcia (anglicky \textit{one way function}), je taká hashovacia funkcia pre ktorú platia nasledujíce tri vlastnosti:
\begin{enumerate}
	\item Pre daný hash $x$ je výpočetne nezvládnuteľné nájsť správu takú, že $ h(x) = y $. Anglicky voláme túto vlastnosť \textit{first preimage resistant}.
	\item Pre danú správu je výpočetne nezvládnuteľné nájsť inú správu s rovnakým hashom. Anglicky voláme túto vlastnosť \textit{second preimage resistant}.
	\item Pre ľubovoľnú správu je výpočetne nezvládnuteľné nájsť inú správu s rovnakým hashom. Anglicky voláme túto vlastnosť \textit{collision resistant}.
\end{enumerate}

Hashovacie funkcie majú v oblasti počítačovej bezpečnosti dôležité využitie:
\begin{itemize}
	\item Bezpečné ukladanie hesiel: Digitálna služba neukladá v databáze heslo, ale len jeho hash. Pri ukradnutí databázy nedochádza k odhalenie hesiel užívateľov.
	\item Integrita dát: Hashovacia funkcia môže byť použitá na ochranu integrity ľubovoľných dát. Ak spočítate hash veľkého súboru a bezpečne ho uložíte tak ste schopný detekovať, že niekto tento súbor zmenil.
	\item Digitálny podpis: Hashovacia funkcia je kryptografické primitívum potrebné pre vytvorenie digitálneho podpisu.
\end{itemize}
Existuje množstvo hashovacích funkcií. Medzi veľmi známe a používané patrí napríklad MD5 (128 bitový výstup), SHA256 (256 bitový výstup), SHA512 (512 bitový výstup).~\cite{cryptoHandbook, nigelSmartCrypto}

\subsection{Digitálny podpis}\label{subsec:sign}
Moderná kryptografia používa pre zaistenie dôvernosti šifrovanie pomocou tajného kľúča. Pre zašifrovanie a dešifrovanie tajnej správy je potrebná znalosť tajného kľúča. Tento spôsob šifrovania zaisťuje dôvernosť avšak nezaisťuje nepopierateľnosť pretože obe komunikujúce strany poznajú tajný kľúč a teda nie je možné právne dokázať kto správu napísal. Na zaistenie nepopierateľnosti sa používa asymetrické šifrovanie, ktoré používa dvojicu kľúčov:
\begin{itemize}
	\item \textbf{Privátny kľúč} je tajný a pozná ho len odosielateľ správy. Odosielateľ používa tento kľúč na zašifrovanie správy.
	\item \textbf{Verejný kľúč} je dostupný komukoľvek. Ktokoľvek s týmto kľúčom dokáže dešifrovať správu.
\end{itemize}
Tieto dva kľúče tvoria dvojicu prepojenú matematickým spôsobom. Zo znalosti verejného kľúča je výpočetne nezvládnuteľné zistiť privátny kľúč. Zašifrovaná správa nie je dôverná pretože ktokoľvek môže použiť verejný kľúč na jej dešifrovanie. Avšak zašifrovaná správa je nepopierateľne napísaná vlastníkom privátneho kľúča. Tento koncept je základom digitálneho podpisu. Pre väčšiu efektivitu sa nešifruje celá správa ale len jej hash (viď Sekcia~\ref{subsec:hash}). Najznámejšie algoritmy na digitálny podpis sú RSA, DSA, ECDSA.~\cite{cryptoHandbook}

\section{Distribuovaná účtovná kniha}
\textit{Účtovná kniha} (anglicky \textit{ledger}) sa v histórii ľudstva dlhodobo používa na záznam rôznych položiek, najčastejšie peňazí a majetku. Príchod digitalizácie a globalizácie rozšíril tento známy koncept o nové požiadavky.
\textit{Distribuovaná účtovná kniha} (anglicky \textit{distributed ledger}) je všeobecne technológia, ktorá poskytuje databázu zdieľanú naprieč viacerými inštitúciami, krajinami a to typicky verejne. Najtypickejším odvetvím využitia distribuovanej účtovnej knihy je bankovníctvo. Banka poskytuje centralizovanú autoritu, ktorá zabezpečuje bezpečnú manipuláciu s peniazmi klientov.~\cite{dltUkReport}

V roku 2008 bola publikovaná práca~\cite{satoshiBitcoin}, ktorá navrhla \textit{decentralizovanú} distribuovanú účtovnú knihu. Práca navrhla koncept elektronického platobného systému, ktorého bezpečnosť je založená na kryptografickom dôkaze namiesto dôvere v centralizovanú autoritu. \textbf{Blockchain} je decentralizovaná účtovná kniha založená na \textit{peer-to-peer} sieti a asymetrickej kryptografii s digitálnym podpisom (pozri~\ref{sec:crypto}).

\subsection{Integrita blockchainu}
Je dôležité pochopiť, že kryptografické vlastnosti blockchainu (predovšetkým integrita  celého jeho dátového obsahu) nie sú zabezpečené samotnou dátovou štruktúrou. Integrita blockchainu spočíva v nasledujúcich vlastnostiach:
\begin{itemize}
	\item Blockchain je distribuovaný v rozsiahlej decentralizovanej sieti. Mnoho uzlov tejto siete vlastní kopiju aktuálneho blockchainu. Validný blockchain je ten ktorý vlastní väčšina siete.
	\item Obsah blockchainu sa dá rozširovať len tak, že sa v danej sieti vytvorí všesmerové vysielanie (anglicky \textit{broadcast}) s návrhovaným novým blokom. Dátová štruktúra blockchainu umožnuje overiť či je dané rozšírenie kryptograficky validné. Ak nie je, žiaden uzol navrhovanú zmenu neakceptuje. Ak je validné a majorita siete zmenu akceptuje (tzv. konsenzus) tak je daný blok pridaný na koniec blockchainu.
\end{itemize}
Blockchain teda zabezpečuje integritu pomocou kryptografického dôkazu a konsenzu v sieti.~\cite{horizenAcademy}

\subsection{Aplikačné využitie}

Blockchain bol navrhnutý a po prvýkrát implementovaný s účelom poskytnúť peňažnú menu nezávislú od centralizovaného bankovníctva. Tento prvý, a najznámejší, blockchain je Bitcoin~\cite{satoshiBitcoin}. Avšak vlastnosti blockchainovej technológie nachádzajú uplatnenie vo veľkom množstve odvetví. Nasledujúci zoznam vymenúva niekoľko aplikácií, ktoré blockchain môže riešiť~\cite{homoliakBlockchain}:

\begin{itemize}
	\item \textbf{Elektronická peňaženka}: Elektronické peňaženky pre obchod s nejakou formou peňazí (typicky v podobe tokenov). Takéto tokeny sú typicky vlastnené pomocou privátneho kľúča, ktorý má uschovaný majiteľ. Majiteľ môže vlstníctvo tokenov presúvať na iné subjekty v danej sieti.
	\item \textbf{Zmenárne}: V dnešnej dobe existuje veľké množstvo kryptomien. S toho dôvodu sa prirodzene zvyšuje dopyt po zmenárni medzi jednotlivými kryptomenami. Klasická zmenáreň je riešená tradične centralizovanou autoritou. Avšak blockchain je vhodnou technológie aj pre decentralizovanú zmenáreň.
	\item \textbf{Súborové systémy}: V dnešnej dobe už existujú decentralizované súborové systémy založené na peer-to-peer sieťach. Implementácia takéhoto decentralizovaného súborového systému ako blockchain by nám umožnila nepopierateľne a trasovateľne verzovať zmeny v obsahu.
	\item \textbf{Správa identít}: Správa identít je typicky centrálna autorita, ktorá prideľuje pre konkrétne entity určité zdroje na ktoré majú právo. Ide o schému podobnú banke. Blockchain by v tomto prípade opäť umožnil náhradu tejto centralizovanej autority za decentralizovanú sieť.
	\item \textbf{Voľby}: Elektronické voľby sú ďalším vhodným príkladom, kde sa dá efektívne využiť blockchain. Voliace entity predstavujú decentralizovanú sieť a vlastnosti blockchainu zase poskytujú transparentnosť a verejnú overiteľnosť.
	\item \textbf{Reputačné systémy}: Reputačné systémy slúžia na meranie úrovne dôvery v určité entity. Typickým príkladom je reputácia rôznych predajcov na základe hlasovania zákazníkov. Transparentnosť a nemennosť blockchainovej histórie by znížila možnosť manipulácie s reputáciu v prospech nejakej entity.
	\item \textbf{Aukcie}: Elektronická aukcia je služba veľmi podobná elektronickej peňaženke alebo zmenárni s podobnými bezpečnostnými požiadavkami. Tieto vlastnosti by opäť dokázala pokryť technológia blockchain.
\end{itemize}

\subsection{Peer-to-peer sieť}
peer-to-peer siete
\subsection{Consensus}
\subsection{Permissioned vs Permissionless}

\section{Datová štruktúra blockchain}
Blockchain je decentralizovaná distribuovaná účtovná kniha. Blockchain je dátová štruktúra podobná zoznamu (anglicky \textit{linked list}). Blockchain organizuje dáta do podmnožín, ktoré sa volajú bloky. Blok je podobný uzlu v zozname. Každý blok obsahuje referenciu na ďalší blok. Rozdiel medzi zoznamom a blockchainom je v tom, že referencia blockchainu je zabezpečená proti manipulácia (anglicky \textit{tamper-evident}) pomocou modernej kryptografie. Bežný zoznam používa referenciu pomocou ukazovateľov (anglicky \textit{pointers}), ktoré môže ktokoľvek a kedykoľvek pozmeniť bez toho aby pozmenil dátový obsah. Naopak, blockchain vôbec neumožňuje meniť už pridané bloky. Jediná povolená operácia je pridanie ďalšieho bloku na koniec blockchainu.~\cite{horizenAcademy}

Vlastnosti blockchainu:
\begin{itemize}
	\item \textbf{Decentralizácia}: Blockchain je peer-to-peer sieť, ktorá nepotrebuje centralizovanú dôveryhodnú autoritu.
	\item \textbf{Perzistencia}: Ak je blockchain dostatočne dlhý a používaný veľkým počtom uživateľov, tak je história je vpodstate nemožné zmeniť históriu blockchainu. 
	\item \textbf{Anonymita}: Uživatelia pracujúci s blockchainom používajú na identifikáciu modernú kryptografiu (asymetrické kľúče, hashovanie a digitálny podpis). Takýto kryptografický identifikátor neodhaľuje skutočnú identitu uživateľa.
	\item \textbf{Auditovateľnosť}: Blockchain v sebe nesie celú histórie zmien jeho obsahu a teda každú zmena stavu dát uložených v blockchaine je možné sledovať.
\end{itemize}

\subsection{Blok}

\subsection{Hlavička bloku}

\subsection{Obsah bloku}


\section{Transakcia}
\section{Markledown strom}



%===============================================================================
