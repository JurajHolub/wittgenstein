% Tento soubor nahraďte vlastním souborem s obsahem práce.
%=========================================================================
% Autoři: Michal Bidlo, Bohuslav Křena, Jaroslav Dytrych, Petr Veigend a Adam Herout 2019
\chapter{Úvod}
TODO

\chapter{Blockchain}\label{chap:blockchain}

\section{Distribuovaná účtovná kniha}
\textit{Účtovná kniha} (anglicky \textit{ledger}) sa v histórii ľudstva dlhodobo používa na záznam rôznych položiek, najčastejšie peňazí a majetku. Príchod digitalizácie a globalizácie rozšíril tento známy koncept o nové požiadavky.
\textit{Distribuovaná účtovná kniha} (anglicky \textit{distributed ledger}) je všeobecne technológia, ktorá poskytuje databázu zdieľanú naprieč viacerými inštitúciami, krajinami a to typicky verejne. Najtypickejším odvetvím využitia distribuovanej účtovnej knihy je bankovníctvo. Banka poskytuje centralizovanú autoritu, ktorá zabezpečuje bezpečnú manipuláciu s peniazmi klientov.~\cite{dltUkReport}

V roku 2008 bola publikovaná práca~\cite{satoshiBitcoin}, ktorá navrhla \textit{decentralizovanú} distribuovanú účtovnú knihu. Práca navrhla koncept elektronického platobného systému, ktorého bezpečnosť je založená na kryptografickom dôkaze namiesto dôvere v centralizovanú autoritu. \textbf{Blockchain} je decentralizovaná účtovná kniha založená na \textit{peer-to-peer} sieti a asymetrickej kryptografii s digitálnym podpisom (pozri~\ref{sec:crypto}).

\subsection{Vlastnosti blockchainu}
Blockchain je datová štruktúra, ktorá má nasledujúce vlastnosti:
\begin{itemize}
	\item \textbf{Decentralizácia}: Blockchain je peer-to-peer sieť, ktorá nepotrebuje centralizovanú dôveryhodnú autoritu.
	\item \textbf{Auditovateľnosť}: Blockchain v sebe nesie celú histórie zmien jeho obsahu a teda každú zmena stavu dát uložených v blockchaine je možné sledovať.
	\item \textbf{Nemennosť}: Pri správnom použití a dostatočne veľkej sieti nie je možné zmeniť histórie alebo dátový obsah blockchainu.
	\item \textbf{Anonymita}: Uživatelia pracujúci s blockchainom používajú na identifikáciu modernú kryptografiu (asymetrické kľúče, hashovanie a digitálny podpis). Takýto kryptografický identifikátor neodhaľuje skutočnú identitu uživateľa.
\end{itemize}

Nemennosť a auditovateľnosť blockchainu sú zabezpečené pomocou samotnej datovej štruktúry, ktorá využíva modernú kryptografiu (viď sekcia~\ref{sec:data-struct-blockchain}) a taktiež pomocou samotnej decentralizovanej siete (viď sekcia~\ref{sec:p2p}).~\cite{horizenAcademy}

\subsection{Aplikačné využitie}

Blockchain bol navrhnutý a po prvýkrát implementovaný s účelom poskytnúť peňažnú menu nezávislú od centralizovaného bankovníctva. Tento prvý, a najznámejší, blockchain je Bitcoin~\cite{satoshiBitcoin}. Avšak vlastnosti blockchainovej technológie nachádzajú uplatnenie vo veľkom množstve odvetví. Nasledujúci zoznam vymenúva niekoľko aplikácií, ktoré blockchain môže riešiť~\cite{homoliakBlockchain}:

\begin{itemize}
	\item \textbf{Elektronická peňaženka}: Elektronické peňaženky pre obchod s nejakou formou peňazí (typicky v podobe tokenov). Takéto tokeny sú typicky vlastnené pomocou privátneho kľúča, ktorý má uschovaný majiteľ. Majiteľ môže vlstníctvo tokenov presúvať na iné subjekty v danej sieti.
	\item \textbf{Zmenárne}: V dnešnej dobe existuje veľké množstvo kryptomien. S toho dôvodu sa prirodzene zvyšuje dopyt po zmenárni medzi jednotlivými kryptomenami. Klasická zmenáreň je riešená tradične centralizovanou autoritou. Avšak blockchain je vhodnou technológie aj pre decentralizovanú zmenáreň.
	\item \textbf{Súborové systémy}: V dnešnej dobe už existujú decentralizované súborové systémy založené na peer-to-peer sieťach. Implementácia takéhoto decentralizovaného súborového systému ako blockchain by nám umožnila nepopierateľne a trasovateľne verzovať zmeny v obsahu.
	\item \textbf{Správa identít}: Správa identít je typicky centrálna autorita, ktorá prideľuje pre konkrétne entity určité zdroje na ktoré majú právo. Ide o schému podobnú banke. Blockchain by v tomto prípade opäť umožnil náhradu tejto centralizovanej autority za decentralizovanú sieť.
	\item \textbf{Voľby}: Elektronické voľby sú ďalším vhodným príkladom, kde sa dá efektívne využiť blockchain. Voliace entity predstavujú decentralizovanú sieť a vlastnosti blockchainu zase poskytujú transparentnosť a verejnú overiteľnosť.
	\item \textbf{Reputačné systémy}: Reputačné systémy slúžia na meranie úrovne dôvery v určité entity. Typickým príkladom je reputácia rôznych predajcov na základe hlasovania zákazníkov. Transparentnosť a nemennosť blockchainovej histórie by znížila možnosť manipulácie s reputáciu v prospech nejakej entity.
	\item \textbf{Aukcie}: Elektronická aukcia je služba veľmi podobná elektronickej peňaženke alebo zmenárni s podobnými bezpečnostnými požiadavkami. Tieto vlastnosti by opäť dokázala pokryť technológia blockchain.
\end{itemize}

\section{Moderná kryptografia}\label{sec:crypto}
Pre pochopenie technológie blockchain je potrebná základná znalosť modernej kryptografie. V tejto sekcii je popísaný kryptografický hash (pozri~\ref{subsec:hash}) a digitálny podpis (pozri~\ref{subsec:sign}). Obe tieto kryptografické primitíva sú základom na ktorom stojí nemennosť, auditovateľnosť a anonymita blockchainu.

\subsection{Hash}\label{subsec:hash}
Hashovacia funkcia je taká funkcia $h$, ktorá má ako parameter $x$ reťazec bitov ľubovolnej dĺžky a vracia reťazec $y$ s konštantnou dĺžkou. Reťazec $y$ voláme hash. Hashovacia funkcia vracia pre konkrétny vstup vždy rovnaký hash.
$$ h(x) = y $$
Kryptografická hashovacia funkcia, alebo tiež jednocestná funkcia (anglicky \textit{one way function}), je taká hashovacia funkcia pre ktorú platia nasledujíce tri vlastnosti:
\begin{enumerate}
	\item Pre daný hash $x$ je výpočetne nezvládnuteľné nájsť správu takú, že $ h(x) = y $. Anglicky voláme túto vlastnosť \textit{first preimage resistant}.
	\item Pre danú správu je výpočetne nezvládnuteľné nájsť inú správu s rovnakým hashom. Anglicky voláme túto vlastnosť \textit{second preimage resistant}.
	\item Pre ľubovoľnú správu je výpočetne nezvládnuteľné nájsť inú správu s rovnakým hashom. Anglicky voláme túto vlastnosť \textit{collision resistant}.
\end{enumerate}

Hashovacie funkcie majú v oblasti počítačovej bezpečnosti dôležité využitie:
\begin{itemize}
	\item Bezpečné ukladanie hesiel: Digitálna služba neukladá v databáze heslo, ale len jeho hash. Pri ukradnutí databázy nedochádza k odhalenie hesiel užívateľov.
	\item Integrita dát: Hashovacia funkcia môže byť použitá na ochranu integrity ľubovoľných dát. Ak spočítate hash veľkého súboru a bezpečne ho uložíte tak ste schopný detekovať, že niekto tento súbor zmenil.
	\item Digitálny podpis: Hashovacia funkcia je kryptografické primitívum potrebné pre vytvorenie digitálneho podpisu.
\end{itemize}
Existuje množstvo hashovacích funkcií. Medzi veľmi známe a používané patrí napríklad MD5 (128 bitový výstup), SHA256 (256 bitový výstup), SHA512 (512 bitový výstup).~\cite{cryptoHandbook, nigelSmartCrypto}

\subsection{Digitálny podpis}\label{subsec:sign}
Moderná kryptografia používa pre zaistenie dôvernosti šifrovanie pomocou tajného kľúča. Pre zašifrovanie a dešifrovanie tajnej správy je potrebná znalosť tajného kľúča. Tento spôsob šifrovania zaisťuje dôvernosť avšak nezaisťuje nepopierateľnosť pretože obe komunikujúce strany poznajú tajný kľúč a teda nie je možné právne dokázať kto správu napísal. Na zaistenie nepopierateľnosti sa používa asymetrické šifrovanie, ktoré používa dvojicu kľúčov:
\begin{itemize}
	\item \textbf{Privátny kľúč} je tajný a pozná ho len odosielateľ správy. Odosielateľ používa tento kľúč na zašifrovanie správy.
	\item \textbf{Verejný kľúč} je dostupný komukoľvek. Ktokoľvek s týmto kľúčom dokáže dešifrovať správu.
\end{itemize}
Tieto dva kľúče tvoria dvojicu prepojenú matematickým spôsobom. Zo znalosti verejného kľúča je výpočetne nezvládnuteľné zistiť privátny kľúč. Zašifrovaná správa nie je dôverná pretože ktokoľvek môže použiť verejný kľúč na jej dešifrovanie. Avšak zašifrovaná správa je nepopierateľne napísaná vlastníkom privátneho kľúča. Tento koncept je základom digitálneho podpisu. Pre väčšiu efektivitu sa nešifruje celá správa ale len jej hash (viď Sekcia~\ref{subsec:hash}). Najznámejšie algoritmy na digitálny podpis sú RSA, DSA, ECDSA.~\cite{cryptoHandbook}


\section{Peer-to-peer sieť}\label{sec:p2p}

Technológia blockchain je postavená na peer-to-peer sieťach. Peer-to-peer sieť sa podieľa na decentralizovanosti, nemennosti a auditovateľnosti blockchainu. 

Peer-to-peer sieť je dynamický súbor nezávislých uzlov (anglicky \textit{peers}), ktoré sú prepojené do grafu. Každý uzol obsahuje zdroje, ktoré zdieľa všetkým ostatným uzlom v sieti.~\cite{p2pBuford, p2pSchollmeier} Dôvod existencie peer-to-peer sietí je teda decentralizovaný spôsob zdieľania zdrojov ako sú súbory, fyzické zariadenia, výpočetný výkon alebo aj elektronické finančné zdroje. Dnes existuje množstvo peer-to-peer sietí. Veľmi známe sú napríklad Gnutella, Kazaa alebo BitTorrent.~\cite{p2pEssence}

\subsection{Referenčný model}
Najbežnejšie technické riešenie peer-to-peer siete je navrstvenie siete (anglicky \textit{overlay network}) na už existujúcu sieť, ktorou je typicky Internet. Takúto sieť potom môžeme definovať ako päticu $(P,R,I,F_P,F_R)$, kde:
\begin{itemize}
	\item $P$ je množina uzlov
	\item $R$ je množinu zdrojov
	\item $I$ je priestor identifikátorov
	\item $F_P: P \rightarrow I$ je funkcia, ktorá mapuje uzoly na identifikátory
	\item $F_R: R \rightarrow I$ je funkcia, ktorá mapuje zdroje na identifikátory
\end{itemize}

Obrázok~\ref{img:p2p-ref-model} ukazuje princíp fungovania takto definovanej siete. Tvorba siete s týmto modelom je potom závislá od šiestich návrhových aspektov:
\begin{enumerate}
	\item Voľba priestoru identifikátorov.
	\item Mapovanie zdrojov a uzlov na identifikátory.
	\item Správa priestoru identifikátorov v réžii uzlov siete.
	\item Tvorba grafu (štruktúra siete).
	\item Stratégia smerovania (anglicky \textit{routing}).
	\item Stratégia údržby.
\end{enumerate}
Konkrétne riešenie pre popisovaných šesť aspektov je zavislé od požiadaviek na efektivitu, škálovateľnosť, samoorganizovateľnosť, odolnosť voči chybám a kooperáciu.~\cite{p2pEssence}

\subsection{Využitie v blockchaine}

Elektronické financie sú typicky reprezentované pomocou elektronických mincí. Takáto minca je reprezentovaná pomocou nejakej sekvencie bitov. Avšak narozdiel od fyzických mincí, elektronické mince umožňujú jednoduchú falzifikáciu. Útočník skopíruje bitový reťazec danej mince a zaplatí ním viacnásobne rôzne produkty. Tento útok sa volá zdvojnásobenia výdavkov (anglicky \textit{double-spending attack}). Proti tomuto útoku existuje tradičné zabezpečenie pomocou centrálnej autority. Banka je centrálna autorita, ktorá schvaľuje všetky manipulácie s elektronickými mincami a teda neumožní použiť mincu takýmto podvodným spôsobom. Avšak toto riešenie nie je možné použiť v decentralizovanej sieti, kde centrálna autorita neexistuje. V prípade decentralizovanej siete je možné zabrániť tomuto útoku pomocou použitia dátovej štruktúry blockchain.~\cite{doubleSpending}

Kryptomena Bitcoin ako prvá navrhla použitie peer-to-peer siete v spojení s blockchain technológiou pre zabránenie double-spending útoku. V takejto sietu je jediný zdroj na zdieľanie a to je dátová štruktúra blockchain v ktorej sú uložené všetky informácie o elektronických financiách. Zjednodušene môžeme povedať, že majorita uzlov siete zdieľa rovnaký zdroj (rovnakú kópiu blockchainu). Ak chce niektorý uzol vykonať finančnú transakciu tak zašle správu s navrhovanou zmenou blockchainu do siete. Uzly v tejto sieti nie je potrebné identifikovať pretože správy posielané v tejto sieti nie sú smerované na žiadne konkrétne miesto. Keď uzol prijme správu s nejakou modifikáciu tak si overí či ide o validnú požiadavku na finančnú transakciu. Štruktúra blockchainu používa modernú kryptografiu na overenie validnosti transakcie (pozri sekciu~\ref{sec:crypto}). Blockchain, ktorý vlastní väčšina siete je ten, ktorý sa považuje za pravdu. Útočník by musel teda vlastniť aspoň 51\,\% uzolov v sieti aby mohol vykonať double-spending útok. Ak je daná sieť dostatočne veľká tak by toho útočník nemal byť schopný dosiahnuť.~\cite{satoshiBitcoin}

\begin{figure}[bt]
	\centering
	\includegraphics[width=\textwidth]{obrazky-figures/p2p-ref-model.pdf}
	\caption{Referenčný model peer-to-peer siete.~\cite{p2pEssence}}
	\label{img:p2p-ref-model}
\end{figure}

\section{Datová štruktúra blockchain}\label{sec:data-struct-blockchain}
Blockchain je dátová štruktúra podobná zoznamu (anglicky \textit{linked list}). Blockchain organizuje dáta do podmnožín, ktoré sa volajú bloky. Blok je podobný uzlu v zozname. Každý blok obsahuje referenciu na ďalší blok. Rozdiel medzi zoznamom a blockchainom je v tom, že referencia blockchainu je zabezpečená proti manipulácia (anglicky \textit{tamper-evident}) pomocou modernej kryptografie. Bežný zoznam používa referenciu pomocou ukazovateľov (anglicky \textit{pointers}), ktoré môže ktokoľvek a kedykoľvek pozmeniť bez toho aby pozmenil dátový obsah. Naopak, blockchain vôbec neumožňuje meniť už pridané bloky. Jediná povolená operácia je pridanie ďalšieho bloku na koniec blockchainu.~\cite{horizenAcademy}

Každý blok obsahuje dáta, ktoré sú typicky vo forme transakcií. Kryptograficky bezpečný blockchain by mohol fungovať aj tak, že v každom bloku bude uložená práve jedna transakcia. Z dôvodu optimalizácie je ale v jednom bloku uložené množstvo transakcií. Vďaka tejto optimalizácii nemusí celá sieť vytvárať konsenzus po každej transakcii. Samotné transakcie v rámci jedného bloku sú ukladané v ďalšej dátovej štruktúre, ktorá taktiež používa kryptografické hashovanie (viď sekcia~\ref{subsec:merkle-tree}).~\cite{narayanan2016bitcoin}

\subsection{Transakcia}
Transakcia je základný prvok blockchainu. Ide o elementárnu dátovú jednotku, ktorá obsahuje dáta uložené v blockchaine. Bitcoin, prvý blockchain, použil transakciu na manipuláciu s elektronickými financiami. Takáto transakcia sa skladá z troch častí:
\begin{itemize}
	\item \textbf{Množina vstupov}: Každý vstup má uložený hash predošlej transakcie s ktorej vychádza. Ďalej definuje, ktoré výstupy s predošlej transakcie si nárokuje. Nakoniec obsahuje digitálny podpis, ktorý autorizuje tvorcu transakcie.
	\item \textbf{Množina výstupov}: Každý výstup má hodnotu, ktorá je uchovávaná v blockchaine (typicky minca nejakej kryptomeny). Suma hodnôt všetkých výstup transakcie musí byť menšia alebo rovná sume všetkých vstupov transakcie. Ak je menšia, tak tento rozdiel je použitý ako odmena pre toho, kto publikoval tento blok blockchainu.
	\item \textbf{Hlavička}: Obsahuje hash transakcie, ktorý je používaný ako unikátny identifikátor pomocou, ktorého sa na transakciu odkazujeme.
\end{itemize}

\subsection{Hlavička bloku}

\subsection{Obsah bloku}
transakcie


\subsection{Binárny hashovací strom}~\label{subsec:merkle-tree}
Binárny hashovací strom alebo tiež Merkle strom (anglicky \textit{Merkle tree}) je datová štruktúra podobná binárnemu stromu, ktorá slúži na efektívne a rýchle vypočítanie hashu veľkého množstva dát. Blockchain používa tento strom na časovo efektívny výpočet hashu všetkých transakcií. Takto vypočítaný hash je uložený v hlavičke bloku.

Merkle strom je vyvážený binarny strom, kde listové uzly obsahujú jednotlivé transakcie uložené v danom bloku blockchainu. Každý nelistový uzol stromu obsahuje hash vypočítaný z jeho potomkov. Koreňový uzol teda obsahuje hash celého stromu a teda aj všetkých transakcií. Pridanie, odobranie, zmena obsahu, alebo zmena poradia transakcií bude teda viesť k zmene koreňového hashu. Konštrukcia stromu, inak povedané výpočet hashu všetkých transakcií, prebieha následovne:
\begin{enumerate}
	\item Všetky transakcie sú uložené do listovej úrovne stromu. Ak je počet transakcií nepárny tak, je posledná vložená dvakrát.
	\item Nad každým listovým uzlom je vypočítaný hash.
	\item Každý nelistový uzol skonkatenuje hash ľavého a pravého syna, vypočíta nad nimi hash a uloží si ho. 
\end{enumerate}
Konštrukcia takéhoto stromu pre $n$ transakcií má časovú zložitosť $O(log(n))$. Takýto spôsob výpočtu hashu je teda veľmi efektívny pre veľké množstvo transakcií (blok v blockchaine bežne obsahuje stovky transakcií).~\cite{merkleTreeBosamia}

Merkle strom umožnuje efektívne šetriť pamäťové nároky blockchainu. Do blockchainu sú neustále pridávané nové bloky, ktoré obsahujú aj rovnaké staré transakcie. Ak už sú transakcie zaznamenané v dostatočne veľkom množstve blokov tak sú z hľadiska bezpečnosti nemenné. V nových blokoch ich už preto nie je potrebné ukladať. Nový blok si preto uloží len hashe starých vetiev stromu, ale ich obsah už nepotrebuje. Takto je zachovaná integrita hashu všetkých transakcií.~\cite{satoshiBitcoin}


%===============================================================================
